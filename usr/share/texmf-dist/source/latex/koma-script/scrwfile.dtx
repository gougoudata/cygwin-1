% \CheckSum{271}
% \iffalse meta-comment
% ======================================================================
% scrwfile.dtx
% Copyright (c) Markus Kohm, 2010-2012
%
% This file is part of the LaTeX2e KOMA-Script bundle.
%
% This work may be distributed and/or modified under the conditions of
% the LaTeX Project Public License, version 1.3c of the license.
% The latest version of this license is in
%   http://www.latex-project.org/lppl.txt
% and version 1.3c or later is part of all distributions of LaTeX
% version 2005/12/01 and of this work.
%
% This work has the LPPL maintenance status "author-maintained".
%
% The Current Maintainer and author of this work is Markus Kohm.
%
% The KOMA-Script bundle consists of all files listed in manifest.txt.
% The work `scrwfile' consists of the files `scrwfile.dtx' and 
% `scrlogo.dtx'.
%
% If you're missing the installation batch file `scrwfile.ins', try
%    tex scrwfile.dtx
% to unpack all files. 
% The package documentation may be produced repeating
%    latex scrwfile.dtx
% at least three times.
% ----------------------------------------------------------------------
% scrwfile.dtx
% Copyright (c) Markus Kohm, 2010-2012
%
% Diese Datei ist Teil des LaTeX2e KOMA-Script-Pakets.
%
% Dieses Werk darf nach den Bedingungen der LaTeX Project Public Lizenz,
% Version 1.3c.
% Die neuste Version dieser Lizenz ist
%   http://www.latex-project.org/lppl.txt
% und Version 1.3c ist Teil aller Verteilungen von LaTeX
% Version 2005/12/01 und dieses Werks.
%
% Dieses Werk hat den LPPL-Verwaltungs-Status "author-maintained"
% (allein durch den Autor verwaltet).
%
% Der Aktuelle Verwalter und Autor dieses Werkes ist Markus Kohm.
%
% Das KOMA-Script-Paket besteht aus allen Dateien, die in manifest.txt
% genannt sind.
% Das Werk `scrwfile' besteht aus den Dateien `scrwfile.dtx' und
% `scrlogo.dtx'.
%
% Falls Sie die Installations-Batch-Datei `scrwfile.ins' vermissen,
% probieren Sie einfach einmal
%    tex scrwfile.dtx
% um alle Dateien auszupacken. Zur englischen Anleitung siehe den
% englischen Kommentar oben.
% ======================================================================
% \fi
%
% \CharacterTable
%  {Upper-case    \A\B\C\D\E\F\G\H\I\J\K\L\M\N\O\P\Q\R\S\T\U\V\W\X\Y\Z
%   Lower-case    \a\b\c\d\e\f\g\h\i\j\k\l\m\n\o\p\q\r\s\t\u\v\w\x\y\z
%   Digits        \0\1\2\3\4\5\6\7\8\9
%   Exclamation   \!     Double quote  \"     Hash (number) \#
%   Dollar        \$     Percent       \%     Ampersand     \&
%   Acute accent  \'     Left paren    \(     Right paren   \)
%   Asterisk      \*     Plus          \+     Comma         \,
%   Minus         \-     Point         \.     Solidus       \/
%   Colon         \:     Semicolon     \;     Less than     \<
%   Equals        \=     Greater than  \>     Question mark \?
%   Commercial at \@     Left bracket  \[     Backslash     \\
%   Right bracket \]     Circumflex    \^     Underscore    \_
%   Grave accent  \`     Left brace    \{     Vertical bar  \|
%   Right brace   \}     Tilde         \~}
%
% \iffalse
%%% From File: scrwfile.dtx
%<*dtx>
\def\LaTeXformat{LaTeX2e}
\ifx\fmtname\LaTeXformat
  \makeatletter
    \let\saved@@end\@@end
    \def\@@end{\csname fi\endcsname\saved@@end}
  \makeatother
  \ProvidesFile{scrwfile.dtx}
%</dtx>
%<(package&identify)|driver>\NeedsTeXFormat{LaTeX2e}[1995/06/01]
%<package&identify>\ProvidesPackage{scrwfile}
%<driver>\ProvidesFile{scrwfile.drv}
%<doc>\ProvidesFile{scrwfile.tex}
%<*dtx|(package&identify)|driver|doc>
  [2012/03/05 v0.1d-alpha LaTeX2e KOMA-Script package (write and clone files)]
%</dtx|(package&identify)|driver|doc>
%<*driver>
  \documentclass{scrdoc}
  \usepackage[ngerman,english]{babel}
  \usepackage{listings}
  \lstnewenvironment{lstcode}[1][]{%
    \lstset{language=[LaTeX]TeX,basicstyle=\ttfamily\small,#1}%
  }{}

  \CodelineIndex
  \RecordChanges
  \GetFileInfo{scrwfile.dtx}
  \title{The \KOMAScript{} package \texttt{scrwfile}%
    \footnote{This is version \fileversion\ of file \texttt{\filename}.}}
  \date{\filedate}
  \author{Markus Kohm}
  
  \newenvironment{Explain}{\par}{\par}
  \let\Macro\cs
  \let\Package\textsf
  \let\File\texttt
  \let\Option\texttt
  \let\Counter\texttt
  \let\ShowOutput\quote
  \let\endShowOutput\endquote
  \let\Parameter\marg
  \let\OParameter\oarg
  \providecommand\PParameter[1]{\mbox{\texttt{\{#1\}}}}
  \let\PName\meta
  \let\PValue\texttt
  \providecommand*{\autoref}[1]{\expandafter\AUTOREF#1:}
  \providecommand*{\AUTOREF}{}
  \makeatletter
  \def\AUTOREF#1:#2:{%
    \edef\@tempa{#1}%
    \edef\@tempb{tab}\ifx\@tempa\@tempb table~\fi
    \edef\@tempb{sec}\ifx\@tempa\@tempb section~\fi
    \ref{#1:#2}%
  }
  \providecommand*{\IndexCmd}[2][]{}
  \providecommand*{\textnote}[2][]{}
  \makeatother

  \begin{document}
  \maketitle
  \DocInput{\filename}
  \end{document}
%</driver>
%<*dtx>
\fi
% Now the installation driver:
%</dtx>
%<*insfile>
\expandafter\let\expandafter\ifbeta\csname iffalse\endcsname
\def\batchfile{scrwfile.dtx}
\input docstrip.tex
\ifToplevel{%
  \Msg{**********************************************************************}
  \Msg{*}
  \Msg{* KOMA-Script presents scrwfile}
  \Msg{* a package to define, configure and several different styles for}
  \Msg{* table of contents, list of floats and the like.}
  \Msg{*}
  \Msg{* This is `\batchfile', a batch file to unpack the package scrwfile,}
  \Msg{* the documentation of the package, and an archive of all these files.}
  \Msg{*}
  \Msg{**********************************************************************}
  \keepsilent
  \askforoverwritefalse
}

\preamble

Copyright (c) 2010-2012 by Markus Kohm <komascript(at)gmx.info>

This file was generated from file(s) of the KOMA-Script bundle.
---------------------------------------------------------------

This work may be distributed and/or modified under the conditions of
the LaTeX Project Public License, version 1.3c of the license.
The latest version of this license is in
  http://www.latex-project.org/lppl.txt
and version 1.3c or later is part of all distributions of LaTeX
version 2005/12/01 or later and of this work.

This work has the LPPL maintenance status "author-maintained".

The Current Maintainer and author of this work is Markus Kohm.

This file may only be distributed together with the file
`scrwfile.dtx' and `scrlogo.dtx'. You may however distribute the files
`scrwfile.dtx' and `scrlogo.dtx' without this file.

If this file is a beta version, you are not allowed to distribute it.

English and German manuals are part of KOMA-Script bundle.
----------------------------------------------------------

The english manual is at `scrwfile.dtx', too.

The KOMA-Script bundle (but not this file) was based upon the LaTeX2.09 
Script family created by Frank Neukam 1993 and the LaTeX2e standard 
classes created by The LaTeX3 Project 1994-1996.

THIS IS AN ALPHA VERSION!
USAGE OF THIS VERSION IS ON YOUR OWN RISK!
EVERYTHING MAY HAPPEN!
EVERYTHING MAY CHANGE IN FUTURE!
THERE IS NO SUPPORT, IF YOU USE THIS PACKAGE!

\endpreamble

\generate{\usepreamble\defaultpreamble
  \file{scrwfile.sty}{%
    \ifx\ifbeta\iftrue\from{scrbeta.dtx}{package,scrwfile}\fi
    \from{scrwfile.dtx}{package,trace,scrwfile,identify,option,body}%
    \from{scrlogo.dtx}{logo}%
  }%
}

\ifToplevel{%
  \generate{\usepreamble\defaultpreamble
    \file{scrwfile.ins}{%
      \from{scrwfile.dtx}{insfile}%
    }%
    \file{scrwfile.tex}{%
      \from{scrwfile.dtx}{doc}%
    }%
    \file{scrwfile.drv}{%
      \from{scrwfile.dtx}{driver}%
    }%
  }%
}

\ifToplevel{%
  \Msg{**********************************************************************}
  \Msg{*}
  \ifbeta
  \Msg{* THIS IS AN ALPHA OR BETA VERSION. YOU SHOULD NOT INSTALL OR USE IT!}
  \Msg{* THERE MAY BE A LOT OF BUGS AT THIS VERSION!}
  \Msg{* PLEASE INSTALL THE RELEASE YOU MAY FIND AT CTAN OR BERLIOS!}
  \else
  \Msg{* To finish the installation you have to copy the file `scrwfile.sty'}
  \Msg{* to folder `tex/latex/scrwfile/' of one of your TEXMF trees.}
  \Msg{* You should also produce the documentation using}
  \Msg{*\space\space latex scrwfile.dtx}
  \Msg{* and copy it to folder `doc/latex/scrwfile/' of one of your}
  \Msg{* TEXMF trees.}
  \Msg{*}
  \Msg{* See the manual of your TeX distribution for more informations about}
  \Msg{* package installation.}
  \fi
  \Msg{*}
  \Msg{**********************************************************************}
}

%</insfile>
%<*dtx|doc>
%<*dtx>
\csname endinput\endcsname
%</dtx>
% \fi
%
% \selectlanguage{english}
%
% \changes{v0.1}{2010/10/01}{start of new package}
%
%\iffalse^^A meta comment
\chapter[{Spare and Replace Files Using \Package{scrwfile}}]
{Spare and Replace Files Using \Package{scrwfile}%
%  \footnote{This chapter has been generated from \File{scrwfile.dtx}.}%
}
\labelbase{scrwfile}

\BeginIndex{Package}{scrwfile}

%\fi^^A meta comment
% \begin{abstract}
\TeX{} supports 18 write handles only. Handle 0 is used by \TeX{} itself (log
file). \LaTeX{} needs at least handle 1 for \Macro{@mainaux}, handle 2 for
\Macro{@partaux}, one handle for \Macro{tableofcontents}, one handle for
\Macro{listoffigures}, one handle for \Macro{listoftables}, one handle for
\Macro{makeindex}. So there are 11 left. Seams a lot and enough. But every new
type of float, every new index and several other packages, e.g.,
\Package{hyperref} need write handles, too. 

The bottom line is, that this eventually will result in the error message:
\begin{lstcode}
  ! No room for a new \write .
  \ch@ck ...\else \errmessage {No room for a new #3}
                                                    \fi 
\end{lstcode}

Last but not least, immediately opening one more write handle for every table
of contents, list of figures, list of tables, and so on has one more
disadvantage. These are not only set by the corresponding commands, they also
couldn't not be set once more, because their helper files are empty after the
corresponding commands until the end of the document.

Package \Package{scrwfile} provides a general change of the \LaTeX{} kernel,
that may solve both problems.
% \end{abstract}
%
% \tableofcontents

\section{General Modifications of the \LaTeX{} Kernel}
\label{sec:scrwfile.kernelpatches}

\LaTeX{} classes use the \LaTeX{} kernel command
\Macro{@starttoc}\IndexCmd{@starttoc} for output and opening of a new table of
contents, list of float, e.\,g., \Macro{listoffigures} or
\Macro{listoftables}, or similar. Thereby \LaTeX{} not only reads the helper
file with the contents of the directory, but opens the helper file for writing
also. Nevertheless, if afterwards new entries to these lists of floats will be
made using \Macro{addcontentsline}, then this will not write immediately to
the helper file. Instead of \LaTeX{} writes
\Macro{@writefile}\IndexCmd{@writefile} commands into the
\File{aux}-file. Only while reading the \File{aux}-file while the end of the
document, those \Macro{@writefile} commands generates real write operations
into the helper files. \LaTeX{} even doesn't close the helper files
explicitly. Instead of \LaTeX{} counts on \TeX{} to close all open files at
the end.

This procedure means, that all the helper files are open for writing from the
output of the contents, i.\,e. at the front matter of the document until the
end of the document, though their content is written after the end of the
document. \Package{scrwfile}'s basic approach is exactly this: redefinition of
\Macro{@starttoc} and \Macro{@writefile}.

Surely\textnote{Attention!}, changes of the \LaTeX{} kernel always may
potentially result in incompatibilities with other packages. In case of
\Package{scrwfile} this may be happen with all packages, that redefine
\Macro{@starttoc} or \Macro{@writefile} too. Sometimes changing the order of
loading the packages may help.

In fact, such problems with \Package{scrwfile} haven't been reported currently,
though several users have tested the package for more than a year until first
release. Nevertheless, if you find such a problem, please contact the
\KOMAScript{} author.


\section{The Single File Feature}
\label{sec:scrwfile.singlefilefeature}

Just while loading the package using, e.\,g.,
\begin{lstcode}[belowskip=\dp\strutbox]
  \usepackage{scrwfile}
\end{lstcode}
\Package{scrwfile} will redefine \Macro{@starttoc}\IndexCmd{@starttoc} to not
longer allocate a file handle or open any file for writing. Immediately before
closing the \File{aux}-file in \Macro{end}\PParameter{document} it will
redefine \Macro{@writefile} to not longer write into the usual helper files
but into one single new file with file extension \File{wrt}. After
reading the \File{aux}-file this \File{wrt}-file will be processed once per
helper file. This means, that not all of the helper file have to be open at
the same time, but only one at a time. And this single file will be closed
afterwards and the write handle is not longer needed after it is closed. An
internal write handle of \LaTeX{} is used for this. So \Package{scrwfile}
doesn't need any own write handle.

Because of this, even if only one table of contents should be generated, after
loading \Package{scrwfile} one more write file handle will be available,
e.\,g., for bibliographies, indexes, glossaries and similar, that are not
using \Macro{@starttoc}. Additionally the number of tables of contents and
lists of whatever, that use \Macro{@starttoc}\IndexCmd{@starttoc} won't be
limited any longer.


\section{The Clone File Write Feature}
\label{sec:scrwfile.clonefilefeature}

Sometimes it is useful to input a file not only once but several times. As
\Macro{@starttoc} does not open files for writing any longer, this may be
done by simply using \Macro{\@starttoc} several times with the same
extension. But sometimes you may have additional entries in only some of the
content directories. Because of this, \Package{scrwfile} allows to copy all
entries of a file to another file, too. We call this cloning.

%\iffalse^^A meta-comment
\begin{Declaration}
%\fi^^A meta-comment
% \DescribeMacro{\TOCclone}
% The command
  \Macro{TOCclone}%
  \OParameter{heading}\Parameter{source}\Parameter{destination}
%\iffalse^^A meta-comment
\end{Declaration}
\BeginIndex{Cmd}{TOCclone}%
%\fi^^A meta-comment
activates the clone feature for files with extensions \PName{source} and
\PName{destination}. All entries to the file
\Macro{jobname}\File{.}\PName{source} will be added to
\Macro{jobname}\File{.}\PName{destination}, too.

If extension \PName{destination} is a new one, \PName{destination} will be
added to the list of known extensions using the \KOMAScript{} package
\Package{tocbasic}.

If the optional argument \PName{heading} is given, a new list-of macro
\Macro{listof}\PName{destination} is defined. \PName{heading} will be used
as section (or chapter) heading of this list. In this case several
\Package{tocbasic} features of the \PName{source} will be copied to
\PName{destination}, if and only if they have been set up when
\Macro{TOCclone} was used. Feature \PName{nobabel} will always be set,
because the language selection commands are part of the helper file and
would be cloned, too.

% \subparagraph*{How you can use it:} 
%\iffalse^^A meta-comment
\begin{Example}
%\fi^^A meta-comment
Assumed, you want a short table of contents
with only the chapter level but an additional entry with the table of
contents:
\begin{lstcode}
  \usepackage{scrwfile}
  \TOCclone[Short \contentsname]{toc}{stoc}
  \AtBeginDocument{%
    % aux first opened at \begin{document}. So wait until this:
    \addtocontents{toc}{% first toc entry:
      \proctect\addcontentsline{stoc}{% write to Short Contents
        \protect\tableofcontents% Contents
      }%
    }%
  }
  \begin{document}
  \setcounter{tocdepth}{1}% show chapters only
  \listofstoc% Write short table of contents
  \setcounter{tocdepth}{6}% show all levels
  \tableofcontents% Write table of contents
\end{lstcode}
You need at least three \LaTeX{} runs to get a short table of contents and a
detailed table of contents. The detailed one produces an entry at the short
one but this entry will not be part of the detailed table of contents.
%\iffalse^^A meta-comment
\end{Example}
\EndIndex{Cmd}{TOCclone}

\section{State of Development Note}
\label{sec:scrwfile.draft}

Though this package has been tested by several users and even is in
productivity usage several times, development is not finished yet. Because of
this especially internal functionality may be changed in future. Most likely
the package will be extended. Some code for extensions is already in the
package. Nevertheless, currently user haven't make requests for such
extension. So there's currently no user documentation for this.

\EndIndex{Package}{scrwfile}

%\fi^^A meta-comment
 
%\iffalse
%</dtx|doc>
%\fi
%
% \StopEventually{%
%   \clearpage
%   \PrintIndex\PrintChanges}
%
%
% \section{Implementation}
%
% \iffalse
%<*package>
%<*identify>
% \fi
%
%    \begin{macrocode}
\PackageWarningNoLine{scrwfile}{%
  THIS IS AN ALPHA VERSION!\MessageBreak
  USAGE OF THIS VERSION IS ON YOUR OWN RISK!\MessageBreak
  EVERYTHING MAY HAPPEN!\MessageBreak
  EVERYTHING MAY CHANGE IN FUTURE!\MessageBreak
  THERE IS NO SUPPORT, IF YOU USE THIS PACKAGE!\MessageBreak
  Maybe, it would be better not to load this package.%
}
%    \end{macrocode}
%\iffalse
%</identify>
%\fi
%
% \subsection{Options}
% \iffalse
%<*option>
% \fi
%
% Currently we don't need options.
%
% \iffalse
%</option>
% \fi
%
% \subsection{Body}
% \iffalse
%<*body>
% \fi
%
% \subsubsection{Needed Packages}
%
% Package \textsf{scrbase} is needed, because of using several \KOMAScript{}
% basic commands.
%    \begin{macrocode}
\RequirePackage{scrbase}[2010/09/17]
%    \end{macrocode}
%
% Package \textsf{tocbasic} is needed for the lists of cloned TOCs.
%    \begin{macrocode}
\RequirePackage{tocbasic}[2010/10/01]
%    \end{macrocode}
%
% Package \textsf{scrlfile} is needed because of the \texttt{aux} file
% handling and \verb|\protected@immediate@write|.
%    \begin{macrocode}
\RequirePackage{scrlfile}[2010/09/30]
%    \end{macrocode}
%
% \subsubsection{\LaTeX{} Kernel Patchs}
%
% For some features we need to patch \LaTeX{} kernel macros. Those features
% and macros are:
% \begin{description}
% \item[Single handle feature] means, that \LaTeX{} will no longer need a file
%   handle for every help file, but only one for all files. We will patch
%  |\@starttoc| and |\@writefile| to do so.
% \item[Clone file feature] means, that every write to one file may be done to
%   another file, too. We will patch |\@writefile| to do so.
% \end{description}
% Every patch should be minimum invasive, so that files, that are not under
% \textsf{scrwfile}'s control are changed as little as possible.
%
% \begin{macro}{\scrwfile@if@only}
% First of all we check, if the file should be handled by \textsf{scrwfile}.
%    \begin{macrocode}
\newcommand*{\scrwfile@if@only}[1]{%
  \begingroup
    \scr@ifundefinedorrelax{scrwfile@only}{\@tempswatrue}{%
      \@tempswafalse
      \edef\reserved@b{#1}%
      \@for\reserved@a:=\scrwfile@only\do
        {\ifx\reserved@a\reserved@b\@tempswatrue\fi}%
    }%
    \if@tempswa
      \scr@ifundefinedorrelax{scrwfile@never}{}{%
        \edef\reserved@b{#1}%
        \@for\reserved@a:=\scrwfile@only\do
          {\ifx\reserved@a\reserved@b\@tempswafalse\fi}%
      }%
    \fi
  \expandafter\endgroup
  \if@tempswa
    \expandafter\@firstoftwo
  \else
    \expandafter\@secondoftwo
  \fi
}
%    \end{macrocode}
% \end{macro}
%
% \begin{macro}{\scrwfile@starttoc}
% \begin{macro}{\scrwfile@@starttoc}
% This is the internal redefinition of |\@starttoc|. First of all test, if it
% should be used, then use it or not.
%    \begin{macrocode}
\newcommand*{\scrwfile@starttoc}[1]{%
  \scrwfile@if@only{#1}{\scrwfile@@starttoc}{\scrwfile@saved@starttoc}{#1}%
}
\newcommand*{\scrwfile@@starttoc}[1]{%
%<trace>  \typeout{Use my own \string\@starttoc\space for #1}%
  \begingroup
    \if@filesw
      \xdef\scrwfile@writefilelist{\scrwfile@writefilelist,#1}%
    \fi
    \@fileswfalse
    \scrwfile@saved@starttoc{#1}%
  \endgroup
}
%    \end{macrocode}
% \end{macro}
% \end{macro}
%
% \begin{macro}{\scrwfile@writefile}
% \begin{macro}{\scrwfile@@writefile}
% \begin{macro}{\scrwfile@wrtout}
% \begin{macro}{\scrwfile@writefilelist}
% This is the internal redefinition of |\@writefile|. First of all test, if it
% should be used, then use it or not.
%    \begin{macrocode}
\newcommand*{\scrwfile@writefile}[1]{%
  \scrwfile@if@only{#1}{\scrwfile@@writefile}{\scrwfile@saved@writefile}{#1}%
}
\newcommand{\scrwfile@@writefile}[2]{%
%<trace>  \typeout{Use my own \string\@writefile\space for #1}%
  \ifnum\scrwfile@wrtout>0
    \begingroup
      \@temptokena{#2}%
      \immediate\write\scrwfile@wrtout{%
        \string\@writefile{#1}{\the\@temptokena}%
      }%
%    \end{macrocode}
% This was the entry for the real file. But we also may have clone files:
%    \begin{macrocode}
      \scrwfile@process@clones{#1}%
    \endgroup
  \fi
}
\chardef\scrwfile@wrtout\z@
\newcommand*{\scrwfile@writefilelist}{}
%    \end{macrocode}
% \end{macro}
% \end{macro}
% \end{macro}
% \end{macro}
%
% \begin{macro}{\@writefile}
% We have to add the single handle feature and the clone file feature to this.
% \begin{macro}{\scrwfile@saved@writefile}
% This is the original definition, that will be used, if the file is not under
% \texttt{scrwfile}'s control.
%    \begin{macrocode}
\newcommand*{\scrwfile@saved@writefile}{}
\BeforeClosingMainAux{%
  \ifx\scrwfile@writefilelist\@empty\else
    \let\scrwfile@saved@writefile\@writefile
    \let\scrwfile@wrtout\@partaux
    \immediate\openout\scrwfile@wrtout \jobname.wrt
    \let\@writefile\scrwfile@writefile
  \fi
}
\AfterReadingMainAux{%
  \ifx\scrwfile@writefilelist\@empty\else
    \immediate\closeout\scrwfile@wrtout
    \chardef\scrwfile@wrtout\z@
    \begingroup
      \let\@writefile\scrwfile@saved@writefile
      \@for\@currext:=\scrwfile@writefilelist\do{%
        \begingroup
          \ifx\@currext\@empty\else
            \scr@ifundefinedorrelax{tf@\@currext}{%
%<trace>              \typeout{Process extension: `\@currext'}
              \immediate\openout\@partaux \jobname.\@currext
              \expandafter\let\csname tf@\@currext\endcsname\@partaux
              \@input@{\jobname.wrt}%
              \immediate\closeout\@partaux
            }{}%
          \fi
        \endgroup
      }%
    \endgroup
  \fi
}
%    \end{macrocode}
% \end{macro}
% \end{macro}
%
% \begin{macro}{\@starttoc}
% We have to add the single handle feature to this.
% \begin{macro}{\scrwfile@saved@starttoc}
% This is the original definition, that will be used, if the file is not under
% \texttt{scrwfile}'s control.
%    \begin{macrocode}
\newcommand*{\scrwfile@saved@starttoc}{}
\let\scrwfile@saved@starttoc\@starttoc
\let\@starttoc\scrwfile@starttoc
%    \end{macrocode}
% \end{macro}
% \end{macro}
%
%
% \subsubsection{Clone TOC Feature}
%
% \textsf{scrwfile} may clone a TOC, that means, every entry to one file will
% be copied to other files, too.  You must not clone recursively!
%
% \begin{macro}{\scrwfile@process@clones}
%    \begin{macrocode}
\newcommand*{\scrwfile@process@clones}[1]{%
  \scr@ifundefinedorrelax{scrwfile@clone@#1}{}{%
    \begingroup
      \let\@@protect\protect\let\protect\@empty\afterassignment\restore@protect
      \edef\reserved@b{\csname scrwfile@clone@#1\endcsname}%
      \edef\reserved@c{,#1}%
      \@for \reserved@a:=\reserved@b\do {%
        \@tempswatrue
        \@for \reserved@d:=\reserved@c\do {%
          \ifx\reserved@d\reserved@a\@tempswafalse\fi
        }%
        \if@tempswa
%<trace>          \typeout{clone entry from `#1' to `\reserved@a'}%
          \immediate\write\scrwfile@wrtout{%
            \string\@writefile{\reserved@a}{\the\@temptokena}%
          }%
          \edef\reserved@c{\reserved@c,\reserved@a}%
        \fi
      }%
    \endgroup
  }%
}%
%    \end{macrocode}
% \end{macro}
%
% \begin{macro}{\TOCclone}
% Clone the entries from the second (first mandatory) argument TOC to the
% third (second mandatory) argument TOC.  If the first (optional) argument was
% given, define |\listof#3name| to this and also define |\listof#3| and clone
% the toc features \texttt{leveldown}, \texttt{numbered}, \texttt{onecolumn}
% and \texttt{totoc} of |#2| to |#3|. The toc feature \texttt{nobabel} will
% always be set, because the babel entries at TOC |#3| will be cloned from TOC
% |#2|. \textbf{Note:} We use owner \texttt{TOCclone} for all cloned
% extensions.
%    \begin{macrocode}
\newcommand*{\TOCclone}[3][]{%
  \scr@ifundefinedorrelax{scrwfile@clone@#2}{%
    \expandafter\protected@edef\csname scrwfile@clone@#2\endcsname{%
      #3,\protect\csname scrwfile@clone@#3\endcsname
    }%
  }{%
    \edef\reserved@b{\csname scrwfile@clone@#2\endcsname}%
    \expandafter\protected@edef\csname scrwfile@clone@#2\endcsname{%
      \csname scrwfile@clone@#2\endcsname,%
      #3,%
      \protect\csname scrwfile@clone@#3\endcsname
    }%
  }%
  \scr@ifundefinedorrelax{scrwfile@clone@#3}{%
    \expandafter\let\csname scrwfile@clone@#3\endcsname\@empty
  }{}%
  \ifattoclist{#3}{%
    \PackageWarning{scrwfile}{`#3' already under control of
      tocbasic.\MessageBreak
      Nevertheless features will be set
    }%
  }{%
    \addtotoclist[TOCclone]{#3}%
  }%
  \setuptoc{#3}{nobabel}%
  \ifstr{#1}{}{%
  }{%
    \@namedef{listof#3name}{#1}%
    \@namedef{listof#3}{\listoftoc{#3}}%
    \iftocfeature{#2}{leveldown}{\setuptoc{#3}{leveldown}}{}%
    \iftocfeature{#2}{numbered}{\setuptoc{#3}{numbered}}{}%
    \iftocfeature{#2}{onecolumn}{\setuptoc{#3}{leveldownonecolumn}}{}%  
    \iftocfeature{#2}{totoc}{\setuptoc{#3}{totoc}}{}%
  }%
}
%    \end{macrocode}
% \end{macro}
%
% \iffalse
%</body>
%</package>
% \fi
%
% \Finale
%
\endinput
%
% end of file `scrwfile.dtx'

%%% Local Variables:
%%% mode: doctex
%%% mode: flyspell
%%% ispell-local-dictionary: "english"
%%% TeX-master: t
%%% End:
